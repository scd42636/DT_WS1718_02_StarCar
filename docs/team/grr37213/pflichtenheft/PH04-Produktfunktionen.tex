\section{Produktfunktionen}

\subsection{Benutzerfunktionen}

\begin{itemize}
  \item Der Benutzer kann das Fahrzeug manuell steuern.
	  \begin{itemize}
		\item Via PC/Laptop
		\item Via Fernsteuerung (Gesten/XBox-Controller)
	  \end{itemize}
  \item Der Benutzer kann einen automatischen Raumscan initiieren.
  \item Der Benutzer kann sich die Karte eines gescannten Raumes anzeigen lassen.
  \item Der Benutzer kann einen Fahrbefehl von Punkt A zu Punkt B erteilen.
\end{itemize}

\begin{description}
  \item[/F0101/]
    \textit{Automatischer Start der Benutzeroberfläche:}
	Verbindet der Benutzer das Fahrzeug mit einer von Ihn gewählten Stromquelle, 
	bootet der Raspberry Pi direkt in die Benutzeroberfläche des Fahrzeugs und verhindert so eine falsche Bedienmöglichkeit des Fahrzeugs.
  \item[/F0102/]
    \textit{GUI - Initialisierung des Fahrzeugs:}
	Der Benutzer kann über einen Button das Fahrzeug initialisieren. Das bedeutet im konkreten Fall, 
	dass zunächst ein Serieller Port geöffnet wird und das Inter Board Protocoll (IBC) gestartet wird. 
	Weiterführende Steuerungsmöglichkeiten dürfen dem Benutzer zu diesem Zeitpunkt nicht zu Verfügung stehen.
  \item[/F0103/]
	\textit{GUI - Modusauswahl:}
	Der Benutzer hat die Möglichkeit zwischen zwei Betriebsmodi auszuwählen:
	\begin{itemize}
		\item Uhrsteuerung
		\item Controllersteuerung
	  \end{itemize}
	Zusätzlich muss der Benutzer, ohne einen Modus auszuwählen, die Möglichkeit erhalten, sich die aktuellen Sensorwerte ansehen zu können.
  \item[/F0104/]
  \textit{Neustart der Benutzeroberfläche:}
  Der Benutzer muss über ein Menü die Möglichkeit erhalten, die Benutzeroberfläche neu zu starten.
  Dies ist insbesondere bei Verbindungsproblemen zum Mikrocontroller unabdingbar.
  \item[/F0105/]
  \textit{Beenden des Systems:}
  Der Benutzer muss über ein Menü die Möglichkeit erhalten, die Benutzeroberfläche sowie den Raspberry Pi ordnungsgemäß herunterfahren zu können.
  \item[/F0106/]
  \textit{GUI - Uhrsteuerung:}
  Wählt der Benutzer den Modus Uhrsteuerung, muss diesem zunächst eine kurze Anleitung dargestellt werden, 
  wie er die Uhren anzulegen hat. Hat der Benutzer diese Information verstanden, muss er diese bestätigen.
  Nach der positiven Bestätigung, muss dem Benutzer die Steuerung anhand von Bildern und Animationen verständlich erklärt werden. 
  Zudem muss der Benutzer über einen Button die Möglichkeit gegeben werden, den Raumscan zu starten /F0111/.
  \item[/F0107/]
  \textit{GUI - Controllersteuerung: }
  Wählt der Benutzer den Modus Controllersteuerung, wird dieser aufgefordert, den Controller griffbereit zu halten. 
  Hat der Benutzer diese Information verstanden, muss er diese bestätigen. 
  Nach der positiven Bestätigung, muss dem Benutzer die Steuerung anhand von Bildern und Animationen verständlich erklärt werden. 
  Zudem muss der Benutzer über einen Button die Möglichkeit gegeben werden, den Raumscan zu starten /F0111/.
  \item[/F0108/]
  \textit{GUI - Navigation:}
  Der Benutzer muss jederzeit die Möglichkeit erhalten, zur Modusauswahl  /F0103/ zurückzukehren und einen anderen Modus wählen zu können. 
  Dabei ist die Navigationstiefe in einem Modus unrelevant.
  \item[/F0109/]
  \textit{Darstellung der Sensorwerte:}
  Dem Benutzer muss nach der Wahl, sich die Sensorwerte anzeigen zu lassen, eine Übersicht der vorhandenen Sensoren und deren aktuellen Werte dargestellt werden.
  \item[/F0110/]
  \textit{Fehleranzeige:}
  Dem Benutzer muss eine Fehleranzeige bereitgestellt werden. Diese muss unabhängig von allen Darstellungen und Benutzereingaben jederzeit gut sichtbar sein. 
  Weiterhin müssen dem Benutzer spezifische Details über einem Fehlerfall dargestellt werden.
  \item[/F0111/]
  \textit{GUI-Raumscan:}
  Der Benutzer muss die Möglichkeit erhalten, nach der Wahl eines Modi, den Raumscan zu starten. 
  Während der Raumscan läuft, werden dem Benutzer die Sensordaten dargestellt /F0109/
\end{description}
