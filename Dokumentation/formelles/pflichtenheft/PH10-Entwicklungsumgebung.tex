\section{Entwicklungsumgebung}

Es wird darauf geachtet, dass alle Entwicklungstools kostenlos \textit{(Freeware)} sind.

\subsection{Software}

\begin{itemize}
	\item Plattform
	\begin{itemize}
		\item Linux Mint 18.2 Sonya
		\item Antergos Linux 17.8
		\item Windows 10
    \end{itemize}
	\item Entwicklertools
	\begin{itemize}
			\item Programmiersprachen
			\begin{itemize}
				\item bash
				\item C/C++
				\item python
			\end{itemize}
			\item Dokumentsprachen
			\begin{itemize}
				\item LaTex
			\end{itemize}
			\item Programme/Tools
			\begin{itemize}
				\item gcc/g++
				\item gdb
				\item GNUmake
				\item texlive
				\item LibreOffice
				\item Microsoft Excel
				\item git
			    \item Subversion
			\end{itemize}
           	\item Services
			\begin{itemize}
				\item github.com
			\end{itemize}
	\end{itemize}
\end{itemize}

\subsection{Hardware}

\begin{itemize}
  \item Desktop-Rechner und Laptops
  werden entweder von der Hochschule gestellt oder sind Privatger�te. Die Spezifikation der Ersteren ist weitestgehend unbekannt/irrelevant, die der Zweiteren werden auf Grund der Einhaltung der Privatssph�re der Mitglieder nicht aufgef�hrt.
  \item Raspberry Pi 3 (genaue Version ??)
  Das ein Raspberry PI mit einem vollst�ndigen Linux-Betriebssystem ausgestattet werden kann, ist dieser hier ebenfalls aufgef�hrt. Verschiedene Mitarbeiter des Projektes benutzen einen eigenen Controller zu Hause.
\end{itemize}

\subsection{Orgware}

\begin{itemize}
  \item Microsoft Excel Listen f�r Arbeitszeiten
  \item studentische E-Mail
\end{itemize}
