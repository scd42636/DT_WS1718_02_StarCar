
\section{Serial Protocoll - IBP}
Das \textbf{Inter Board Protocoll (IBP)} dient als Vereinbarung von Kommunikationsregeln über eine Kommunikationsschnittstelle.
Innerhalb dieses Projektes soll es die Kommunikation über eine serielle USB Schnittstelle erleichtern.

\subsection{Konzepte}

\subsection{Aufbau}

\large 
Legende:\\
\normalsize
\begin{center}
\small
\begin{tabular}{l l}
MID &	Identifikationsnummer der Anfrage\\
EID &	Nummer des Fehlers bzgl. MID\\
SIZE&	Größe einer dynamischen Payload, nicht existent bei statisch vereinbarter Übertragung!\\
STAT&	Status der Übertragung;Protokollinterne Fehlererkennung\\
HH	&	Hash des Headers\\
SH	&	Hash der dynamischen Größe (unwichtig bei statisch)\\
DH	&	Hash der Payload\\
Payload & Nutzdaten\\
\end{tabular}
\end{center}
\normalsize

\large 
Request\\
\normalsize

\begin{center}
\begin{bytefield}{16}
	\bitheader{0,7,8,11,12,13,14,15,16}\\
	\begin{leftwordgroup}{Header}
		\bitbox{8}{MID} & \bitbox{4}{STAT} & \bitbox{2}{SH} & \bitbox{2}{HH}
	\end{leftwordgroup}\\
	\bitbox{8}{SIZE} & \bitbox[lt]{8}{}\\
	\wordbox[rlt]{1}{Payload}\\
	\skippedwords\\
	\wordbox[rlb]{1}{}\\
	\begin{leftwordgroup}{Footer}
		\bitbox{8}{DH} & \bitbox[l]{8}{}
	\end{leftwordgroup}\\
\end{bytefield}
\end{center}

\large Response\\
\normalsize
\begin{center}
	\begin{bytefield}{16}
	\bitheader{0,3,4,5,6,7,8,16}\\
	\begin{leftwordgroup}{Header}
		\bitbox{4}{STAT} & \bitbox{2}{SH} & \bitbox{2}{HH} & \bitbox[l]{8}{}
	\end{leftwordgroup}\\
		\bitbox{8}{SIZE} & \bitbox[lb]{8}{}\\
	\wordbox[rl]{1}{Payload}\\
	\skippedwords\\
	\wordbox[rlb]{1}{}\\
	\begin{leftwordgroup}{Footer}
		\bitbox{8}{DH} & \bitbox[l]{8}{}
	\end{leftwordgroup}\\
\end{bytefield}
\end{center}

\large Negative Response\\
\normalsize
\begin{center}
	\begin{bytefield}{16}
	\bitheader{0-63}\\
	\bitbox{4}{status} & \bitbox{2}{sizehash} & \bitbox{2}{hh} & \bitbox{8}{MID} & \bitbox {8}{EID}
\end{bytefield}
\end{center}
